\chapter{Introduzione}

Un sistema di rilevamento di intrusioni, anche detto intrusion detection system (IDS),  è un'applicazione o un dispositivo volta a monitorare continuamente la rete per identificare attività malevole. 

In particolare un IDS controlla il traffico di rete oppure i log di sistema, alla ricerca di possibili anomalie che potrebbero indicare la presenza di attacchi.

Per migliorare il rilevamento delle intrusioni, nell'ultimo decennio, si è iniziato ad utilizzare algoritmi di Machine Learning e Deep Learning attingendo informazioni dai Big Data. \cite{NetworkIntrusionDetection2021}

Questo però ha portato a nuove problematiche, come ad esempio il fatto che i modelli di Machine Learning sono molto sensibili ai dati di addestramento.


Un dataset contenente pacchetti di rete sarà composto per la maggior parte da pacchetti "normali", cioè pacchetti di traffico abituale e, per la restante parte, da pacchetti di attacchi. 
I vari dataset sul traffico di rete presenti oggi però, non hanno una quantità sufficiente di attacchi per poter addestrare al meglio i modelli. 

Un modello addestrato su un dataset con una percentuale di attacchi troppo bassa, non sarà in grado di rilevare bene gli attacchi ~\cite{gopalanBalancingApproachesML2021}.

% In particolare i pacchetti "normali" fanno parte della comune distribuzione dei dati  mentre i pacchetti di attacchi sono fuori da questa distribuzione. Chiameremo quindi i primi, in-distribution (ID) samples, mentre i secondi, verranno chiamati out-of-distribution samples (OOD).
Un metodo per sopperire a questa mancanza è quello di generare nuovi dati a partire da quelli già esistenti (Data Augmentation). I dati generati devono essere però sufficientemente differenti di modo da rappresentare meglio le anomalie che si hanno nella rete durante un attacco.


In questa tesi esploreremo un algoritmo di Data Augmentation chiamato Soft-Brownian-Offset (SBO) \cite{sbo} e lo utilizzeremo per generare nuovi dati per addestrare un IDS.

In particolare Soft-Brownian-Offset permette di generare campioni così detti out-of-distribution (OOD), cioè campioni che non fanno parte della distribuzione dei dati di partenza. Questo è coerente col fatto che di solito i pacchetti di attacchi sono fuori dalla distribuzione dei pacchetti comuni.


