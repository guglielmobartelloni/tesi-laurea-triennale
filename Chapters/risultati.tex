\chapter{Risultati}
\label{chap:risultati}

Di seguito verranno mostrati i più significativi riguardanti la tesi prodotti atraverso del codice Python utilizzando diverse librerie, tra cui:

\begin{itemize}
    \item Pandas e Numpy, per la manipolazione dei dataset
    \item Plotly, per la visualizzazione dei dati
    \item UMAP, per la riduzione della dimensionalità dei dataset 
    \item SBO, per la generazione dei dati fuori dalla distribuzione
    \item XGBoost, il modello addestrato
    \item Scikit-learn, per la valutazione del modello e per il preprocessamento dei dati
\end{itemize}

Tutto il codice utilizzato si trova su \href{https://github.com/guglielmobartelloni/thesis-utils}{Github}~\cite{github} .

\section{Preparazione dei dati}

Come detto nel capitolo \ref{chap:generazione_di_attacchi_usando_sbo}, i dati i dati non possono essere utilizzati come sono ma hanno bisogno di un preprocessamento. In particolare, sono stati eliminati i dati non numerici attraverso il metodo di Pandas "pandas.DataFrame.replace" come segue:


\begin{python}
input_data.replace([np.inf, -np.inf], -1, inplace=True)
input_data.replace(np.nan, -1, inplace=True)
\end{python}

Si estrae poi la colonna label dal dataset, che dovrà essere utilizzata successivamente per il modello, e si procede al one hot encoding dei dati attraverso il medoto "pandas.get\_dummies":

\begin{python}
input_data = pd.get_dummies(input_data)
\end{python}

Infine si cambia le etichette dei vari attacchi in "attack":

\begin{python}
attacks_packets_types = ['Bot', 'DDOS attack-HOIC', 'DDOS attack-LOIC-UDP',
                            'DoS attacks-Hulk', 'DoS attacks-SlowHTTPTest', 'FTP-BruteForce',
                            'Infilteration', 'SSH-Bruteforce']
attacks_packets.replace(attacks_packets_types, 'attack', inplace=True)
\end{python}


\section{Generazione dei pacchetti}

Si è generato i nuovi pacchetti attraverso la libreria python SBO fornita da ~\cite{sbo}. Il metodo "soft\_brownian\_offset" della libreria richiede:

\begin{itemize}
    \item i dati da cui generare i nuovi pacchetti
    \item d\_min ($d^{-}$)
    \item d\_off ($d^{+}$)
    \item softness ($\sigma$)
    \item numero di pacchetti da generare
\end{itemize}

\begin{python}
    data_ood = soft_brownian_offset(data_i, d_min_, d_off_,
                                        softness=softness_,
                                        n_samples=n_ood_samples)
\end{python}



\section{Addestramento del modello}

\section{Calcolo delle metriche}


