\chapter{Generazione di attacchi usando Soft Brownian Offset}

In questo capitolo verranno presentate le tecniche utilizzate per generare i dati fuori dalla distribuzione (Out-of-distribution OOD) per l'addestramento del modello XGBoost. Saranno quindi illustrati i metodi.

% Parleremo inizalmente del processo di preparazione ed analisi dei dati, di come è stato utilizzato Soft Brownian Offset ed infine verranno illustrati i metodi di calcolo dell'accuratezza del modello

Per gestire e processare i dati è stato utilizzato il linguaggio di programmazione Python usando diverse librerie tra le quali possiamo trovare:

\begin{itemize}
    \item Pandas e Numpy, per la manipolazione dei dataset
    \item Plotly, per la visualizzazione dei dati
    \item UMAP, per la riduzione della dimensionalità dei dataset 
    \item SBO, per la generazione dei dati fuori dalla distribuzione
    \item XGBoost, il modello addestrato
    \item Scikit-learn, per la valutazione del modello e per il preprocessamento dei dati
\end{itemize}

Tutto il codice che è stato utilizzato si trova su ~\cite{github}.


\section{Metodologie}


\subsection{Analisi dei dati iniziali}

\subsection{Generazione dei pacchetti}

\subsection{Addestramento del modello}

\subsection{Metriche utilizzate}

\subsubsection{Accuracy}

\subsubsection{Metthews}

